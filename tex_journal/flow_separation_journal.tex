\documentclass[11pt]{article}
\usepackage{geometry}  
\geometry{a4paper} % or letterpaper (US) or a5paper or....
 \geometry{margin=1in} % for example, change the margins to 2 inches all round          
\usepackage{graphicx}
\usepackage{amsmath}
\usepackage{amssymb}
\usepackage{amsthm}
\usepackage{ulem}
\usepackage{cancel}
\usepackage{listings}
\DeclareGraphicsRule{.tif}{png}{.png}{`convert #1 `dirname #1`/`basename #1 .tif`.png}
%\includeonly{Chapter1}

\newcommand{\pdv}[2]{\frac{\partial #1}{\partial #2}}
\newcommand{\mdv}[1]{\frac{\hbox{D} #1}{\hbox{D}t}}
\newcommand{\dv}[2]{\frac{\hbox{d} #1}{\hbox{d} #2}}
\newcommand{\vgr}{\vec{\nabla}}
\newcommand{\tgr}{\vec{\nabla}}
\newcommand{\sech}{\text{sech}}
\newcommand{\eps}{\varepsilon}
\newcommand{\lap}{\nabla^2}
\renewcommand{\Pr}{\text{Pr}}
\renewcommand{\Re}{\text{Re}}
\newcommand{\Ra}{\text{Ra}}
\newcommand{\Ri}{\text{Ri}}
\newcommand{\overbar}{\overline}
\renewcommand{\bar}{\overbar}

\newtheorem{theorem}{Theorem}
\newtheorem{corollary}[theorem]{Corollary}
\newtheorem{definition}{Definition}
\newtheorem{lemma}{Lemma}
\newtheorem{exercise}{Exercise}
\newtheorem{remark}{Remark}
\newtheorem{example}{Example}
\newtheorem{warning}{Warning}

\def\lam{\lambda}
\def\grad{ \mbox{grad}}
\def\curl{ \mbox{curl}}
\def\div{ \mbox{div}}
\def\U{\ensuremath {\cal U}}
\def\S{\ensuremath {\cal S}}
\def\V{\ensuremath {\cal V}}
\def\R{\ensuremath {\cal R}}
\def\tr{\ensuremath {\mbox{tr}}}
\def\eq#1\eq{\begin{align*}#1\end{align*}}
\def\[#1\]{\begin{align}#1\end{align}}

%%% The "real" document content comes below...
\usepackage{color}
 
\definecolor{codegreen}{rgb}{0,0.6,0}
\definecolor{codegray}{rgb}{0.5,0.5,0.5}
\definecolor{codepurple}{rgb}{0.58,0,0.82}
\definecolor{backcolour}{rgb}{0.95,0.95,0.95}

\lstdefinestyle{mystyle}{
    backgroundcolor=\color{backcolour},   
    commentstyle=\color{codegreen},
    keywordstyle=\color{magenta},
    numberstyle=\tiny\color{codegray},
    stringstyle=\color{codepurple},
    basicstyle=\ttfamily\small,
    breakatwhitespace=false,         
    breaklines=true,                 
    captionpos=b,                    
    keepspaces=true,                 
    numbers=left,                    
    numbersep=5pt,                  
    showspaces=false,                
    showstringspaces=false,
    showtabs=false,                  
    tabsize=2
}

\lstset{style=mystyle}
\title{Flow Separation}
\author{Rob Irwin}
%\date{July, 2013} % Activate to display a given date or no date (if empty),
         % otherwise the current date is printed 

\begin{document}

\maketitle

\section{Initializing the MITgcm}

\subsection{Test files}

A standard 2D grid of dimensions $100\times1$km was introduced in the $x$-$z$ plane. Hyperbolic tangent topography was generated which had an extent of $50$m in depth in the west to $1000$m of depth in the east. A linear temperature profile was introduced; the surface was fixed at $14^\circ$C, and the lowest layer was fixed at $2^\circ$C. Boundary conditions were set on the west side for an inflow of $U(x=0,z,t)=1$m/s, $T(x=0,z,t)=14^\circ$C. The flow of $1$m/s would typically reach the shelf break after $1$ day.

\subsection{Adjusting the \texttt{data} parameters for numerical stability}

The MITgcm was run serially on Snapper at varying resolutions on a 2D grid. The data file was initially stripped to the bare essentials and then modified to determine which lines were necessary. Initially, the file consisted simply of:

\begin{lstlisting}
# Model parameters
# Continuous equation parameters
 &PARM01
 tRef=14.,
 sRef=30.,
 no_slip_sides=.TRUE.,
 no_slip_bottom=.TRUE.,
 viscAh=1E-2,
 viscAz=1E-2,
 f0=1E-4,
 beta=0.E-11,
 tAlpha=2.E-4,
 sBeta=7.4E-4,
 gravity=9.81,
 readBinaryPrec=64,
 nonHydrostatic=.FALSE.,
 &
# Elliptic solver parameters
 &PARM02
 &
# Time stepping parameters
 &PARM03
 nIter0=0,
 nTimeSteps=19440,
 deltaT=40,
 dumpFreq=86400.0,
# Gridding parameters
 &PARM04
 delXfile='delX_Nx_256.bin',
 delY=100.,
 delZ=64*15.625,
 &
 PARM05
 bathyFile='Bathy_profile_Nx_256_Nz_64.bin',
 hydrogThetaFile='T_profile_Nx_256_Nz_64.bin',
 uVelInitFile=,
 &
\end{lstlisting}

However on a $256 \times 64$ grid (in $x$ and $z$ respectively) the numerics were unstable and grid-scale oscillations developed in the $u$ field, causing the final solutions to be \texttt{Inf} or \texttt{NaN}. The first attempt to remedy this was to change the lateral eddy viscosity line (denoted by \texttt{viscAh}), and use \textit{Jamart wet points} \cite{jamart}:

\begin{lstlisting}
 viscAh=1E0,
 useJamartWetPoints=.TRUE.,
\end{lstlisting}

Jamart wet points force the code to calculate averages excluding boundary points. This change was insufficient; the output continued to show grid-scale oscillations in the $u$ field. Subsequently, three lines were added:

\begin{lstlisting}
 viscAhGridMin=0.01,
 viscAhGridMax=1.0,
 viscAhReMax=10.0,
\end{lstlisting}

These lines were imported from Karina's MITgcm data initialization file. However, the grid-scale oscillations continued and the output eventually blew-up. 

Smagorinsky eddy viscosity was added as the next attempt.

\begin{lstlisting}
 viscC2Smag=2.2,
\end{lstlisting}

The inclusion of this Smagorinsky eddy viscosity seemed to remedy the issues of grid-scale oscillations and unstable output. To determine which parameters could be disposed of, the previous set of \texttt{viscAh*} settings were reverted to the initial file settings, however the Jamart wet points flag was kept on. The code blew up again, and so the \texttt{viscAh} flag was reverted back to \texttt{viscAh=1E0}.

\bibliographystyle{apalike}

\bibliography{bib}


\end{document}
